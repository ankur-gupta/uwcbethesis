%%
%% UW CBE Thesis: chaptertwoname/chaptertwoname.tex
%% Written by Ankur Gupta, Sep 1, 2013
%% Departament of Chemical and Biological Engineering
%% University of Wisconsin-Madison
%% Copyright (c) Ankur Gupta, 2014
%%
%% License: GPL v3. See LICENSE.
%%

% This is Chapter 2. Good job!
% You're now on a roll.
% Keep going. Let the momentum drive you.
\chapter{Chapter Two Title}
\label{chap:chaptertwoname}
\doublespacing

Now that you're starting a proper chapter, you can make better use of
indexing. Index away the words that are specific to your thesis. Such as,
this is how you index. Radium\index{radium} was discovered in the form of
\index{radium chloride} by Marie Curie and Pierre Curie in 1898. \\

Your job while indexing is to make things easier for the reader.
Unlike a purely printed versions of written texts, a modern PDF
(especially this one) allows for excellent searching using the
``Find'' option of a PDF viewing software. This somewhat lessens the need
for an index --- the reader can simply search for the keyword he or she is
looking for. This means that your index should offer something else.
I find that an index has value when the author has put some effort into it.
For example, if you discuss a particular ``radium'' repeatedly in this chapter
and then again in \chapterfourname{}, then you should index ``radium'' once here and then
once in \chapterfourname{}. Don't index every occurrence of the word ``radium''
in the same chapter (unless your chapter is too long and diffused). The benefit
of such indexing is that you've just saved the reader a lot of time. The reader when
searching for ``radium'' will find that the ``Find'' functionality in the PDF viewer
software finds every, single occurrence of the word. If you've mentioned the word
20 times or more in a chapter then this quickly becomes annoying. On the other hand,
if the reader looks at your index, he or she will find two references to ``radium'' --- one
in this chapter and another in \chapterfourname{}. The reader will thank you. \\

Also, you probably need a lot of references. So, cite away. \\

This chapter is organized as follows. In \mysec{sec:chaptertwoname:sectiononename}, I describe
a lot of cool stuff. In \mysec{sec:chaptertwoname:dragons} there is more cool stuff and
dragons. Who doesn't like dragons? Finally, in \mysec{sec:chaptertwoname:sectionthreename},
I tell you that \emph{Winter is Coming}. \\

\section{Section One Name}
\label{sec:chaptertwoname:sectiononename}
\index{cool stuff!Valyrian Steel}
I promised you cool stuff and here it is --- Valyrian Steel. Use it wisely.

\begin{assump}[Fundamental hypothesis of about swords]
\label{assump:fundamental}
\normalfont
For every sword ever made, there exists a swordsman willing to wield it.
\begin{align}
\forall \text{ Sword}, \exists \text{ Swordsman}
\end{align}
\end{assump}
With this assumption, we proceed to list the famous swords.

\begin{table}
\caption{Famous Swords}
\label{tab:famousswords}
\centering
\begin{tabular}{lll}
\toprule
Sword Name & Intended Swordsman & Story\\
\midrule
Excalibur & King Arthur & Arthurian legend \\
Ice & Eddard Stark & A Song of Fire and Ice \\
And\'{u}ril & Aragorn & Lord of the rings \\
\bottomrule
\end{tabular}
\end{table}


\subsubsection{A Hobbit's sword}
\label{sec:chaptertwoname:hobbitsword}
Many do not consider a hobbit's sword to be a sword. Beware that who ridicules the 
hobbit's dagger because many an orcs have been killed at its edge.


\section{Dragons}
\label{sec:chaptertwoname:dragons}
\subsection{Dragons: What are they?}

\begin{defn}[Dragon]
\normalfont
A \emph{dragon} is a legendary creature, with serpentine or reptilian features.
\end{defn}


\section{Winter is Coming}
\label{sec:chaptertwoname:sectionthreename}

Winter is Coming. I told you. \\

Here is a mathematical equation to help you out.

\begin{align}
q = \epsilon \sigma \left( T_{\text{h}}^4 - T_{\text{c}}^4 \right)
\end{align}

