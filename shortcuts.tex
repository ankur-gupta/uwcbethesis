%%
%% UW CBE Thesis: shortcuts.tex
%% Written by Ankur Gupta, Sep 1, 2013
%% Departament of Chemical and Biological Engineering
%% University of Wisconsin-Madison
%% Copyright (c) Ankur Gupta, 2014
%%
%% License: GPL v3. See LICENSE.
%%


% Shortcuts.tex

% Chapter Shortcuts
% Name shortcuts using names that you can remember
% Example: Use \litsurvey instead of \chaptertwo
% You can use them throughout the thesis like this:
% \litsurvey{} presents a literature survey of the current research methodologies.
\newcommand{\intro}{Chapter~\ref{chap:intro}} % Introduction is Chapter 1
\newcommand{\chaptertwoname}{Chapter~\ref{chap:chaptertwoname}}
\newcommand{\chapterthreename}{Chapter~\ref{chap:chapterthreename}}
\newcommand{\chapterfourname}{Chapter~\ref{chap:chapterfourname}}
\newcommand{\chapterfivename}{Chapter~\ref{chap:chapterfivename}}
\newcommand{\conc}{Chapter~\ref{chap:conclusions}}
\newcommand{\appA}{Appendix~\ref{app:A}}

% Section reference
\newcommand{\mysec}[1]{Section~\ref{#1}}
\newcommand{\mysecs}[2]{Sections~\ref{#1} and~\ref{#2}}

% Equation shortcuts
\newcommand{\myeq}[1]{Eq.~\eqref{#1}}
\newcommand{\myeqr}[2]{Eqs.~\eqref{#1}-\eqref{#2}} % Range
\newcommand{\myeqs}[3]{Eqs.~\eqref{#1},~\eqref{#2},~\eqref{#3}}

% Figures shortcuts
\newcommand{\myfig}[1]{Figure~\ref{#1}}
\newcommand{\myfigr}[2]{Figures~\ref{#1}-\ref{#2}} % Range
\newcommand{\myfigs}[3]{Figures~\ref{#1},~\ref{#2},~\ref{#3}}

% Tables shortcuts
\newcommand{\mytab}[1]{Table~\ref{#1}}
\newcommand{\mytabr}[2]{Tables~\ref{#1}-\ref{#2}} % Range
\newcommand{\mytabs}[3]{Tables~\ref{#1},~\ref{#2},~\ref{#3}}

% Algorithm reference
\newcommand{\myalgo}[1]{Algorithm~\ref{#1}}


% Frequently used abbreviations
\newcommand{\ie}{\emph{i.\,e.}}
\newcommand{\Ie}{\emph{I.\,e.}}
\newcommand{\eg}{\emph{e.\,g.}}
\newcommand{\Eg}{\emph{E.\,g.}}

%%% MATH SHORTCUTS %%%
% Lowercase bolds
\newcommand{\bfx}{\mathbf{x}}
\newcommand{\bfy}{\mathbf{y}}
\newcommand{\bfr}{\mathbf{r}}
\newcommand{\bfz}{\mathbf{z}}
\newcommand{\bfv}{\mathbf{v}}
\newcommand{\bff}{\mathbf{f}}
\newcommand{\bfh}{\mathbf{h}}

% Uuppercase bolds
\newcommand{\bfX}{\mathbf{X}}
\newcommand{\bfY}{\mathbf{Y}}
\newcommand{\bfZ}{\mathbf{Z}}
\newcommand{\bfP}{\mathbf{P}}
\newcommand{\bfQ}{\mathbf{Q}}
\newcommand{\bfR}{\mathbf{R}}
\newcommand{\bfI}{\mathbf{I}}
\newcommand{\bfC}{\mathbf{C}}
\newcommand{\bfV}{\mathbf{V}}
\newcommand{\bfS}{\mathbf{S}}
\newcommand{\bfH}{\mathbf{H}}
\newcommand{\bfzero}{\mathbf{0}}

% Fancy mathcals (for reactions)
\newcommand{\fR}{\mathcal{R}}
\newcommand{\fX}{\mathcal{X}}
\newcommand{\fS}{\mathcal{S}}
\newcommand{\fN}{\mathcal{N}}

% Number systems and sets
\newcommand{\bbR}{\mathbb{R}}
\newcommand{\bbZ}{\mathbb{Z}}
\newcommand{\bbN}{\mathbb{N}}
\newcommand{\bbQ}{\mathbb{Q}}
\newcommand{\bbS}{\mathbb{S}}
\newcommand{\bbA}{\mathbb{A}}
\newcommand{\bbE}{\mathbb{E}}

\newcommand{\abs}[1]{\left| #1 \right|} % cardinality or absolute value

