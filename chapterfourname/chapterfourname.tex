%%
%% UW CBE Thesis: chapterfourname/chapterfourname.tex
%% Written by Ankur Gupta, Sep 1, 2013
%% Departament of Chemical and Biological Engineering
%% University of Wisconsin-Madison
%% Copyright (c) Ankur Gupta, 2014
%%
%% License: GPL v3. See LICENSE.
%%

% This is Chapter 4. Are you more than half way there?
% Keep going.
\chapter{Chapter Four Title}
\label{chap:chapterfourname}
\doublespacing

\section{Citations}
\label{sec:chapterfourname:citations}

Let's talk about citations. Ideally, you want to create a \texttt{.bib} file 
for your thesis. Avoid naming this bib file \texttt{thesis.bib} because it might 
get deleted\footnote{Murphy's Law\index{Murphy's Law} is especially powerful while writing 
a thesis. That's a fact. Always keep a current backup.}.
This thesis template comes with a file named \texttt{citations.bib}. 
Alternatively, you can store all your citations for all your latex projects 
in a common location. In Unix systems, this location is usually at 
\begin{code}
~/texmf/bibtex/bib/
\end{code}


\section{History of \LaTeX}
\label{sec:chapterfourname:history_of_latex}

\LaTeX~\cite{lamport:1986} is a document preparation system based on 
\TeX~\cite{knuth:1999}. \TeX was created by Donald Knuth, who is known 
for the multi-volume work such as~\citet{knuth:1997}.



\section{All about Radium}
\label{sec:chapterfourname:all_about_radium}
\index{radium}

As promised, we talk about ``radium'' again. We discussed ``radium'' 
in \chaptertwoname{} as well. And, as discussed, we need to index ``radium'' 
here again. 


