%%
%% UW CBE Thesis: intro/intro.tex
%% Written by Ankur Gupta, Sep 1, 2013
%% Departament of Chemical and Biological Engineering
%% University of Wisconsin-Madison
%% Copyright (c) Ankur Gupta, 2014
%%
%% License: GPL v3. See LICENSE.
%%


% Begin your journey here.
% People say that one writes the introduction at last, after the conclusions.
% I say that I like to know where I'm going.
% Choose either path but begin now.

\chapter{Introduction}
\doublespacing

\section{Motivation}
\label{sec:intro:motivation}
% Motivate your audience, young Jedi.
% Those in your field of research will most likely skip this.
% You need to write this for those who are not in your field of research.
This is the motivation\index{motivation}. Motivate your audience.
Why is your research important? Why is the reader reading your thesis?
What new conclusions does this thesis gather?

\section{Notation and language}
\label{sec:intro:notation}
Put any guidelines here. If your thesis contains a lot of mathematics,
I suggest you tell your audience about your notational scheme. Did you use the
same symbols throughout your thesis or does the notation differ in every chapter? \\

Things like ``How to read this thesis'' are appropriate here.


\section{An overview of the thesis}
\label{sec:intro:overview}
This dissertation considers the following issues. I hope it enlightens you. \\

\noindent \textbf{\chaptertwoname{} -- Chapter Two Title}
This text provides a summary of what Chapter 2 contains. \\


\noindent \textbf{\chapterthreename{} -- Chapter Three Title}
This text provides a summary of what Chapter 3 contains. \\


\noindent \textbf{\chapterfourname{} -- Chapter Four Title}
This chapter describes how to use citations and provided a brief 
history of \LaTeX. \\

\noindent \textbf{\chapterfivename{} -- Chapter Five Title}
This chapter contains how to use shortcuts and the config file.
\mysec{sec:chapterfivename:defining_shortcuts} describes how to use 
shortcuts. \mysec{sec:chapterfivename:using_config_file} describes 
the use of the config file. \\


\noindent \textbf{\conc{} -- Conclusions and future directions.}
A summary of the major contributions of this dissertation is presented. Specific
areas for improvement for future research are identified. 












